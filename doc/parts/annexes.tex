\chapter{Annexes}\label{annexes}

\section{Au sujet de la langue}

Le code a été délibérément écrit et documenté en langue anglaise. Il n'est point question d'un 
rejet de la langue française: en vérité, je suis parti du principe que la plupart des 
programmeurs ont souvent une bonne maîtrise de la langue anglaise, et que dans l'éventualité où 
le code intéresserait des programmeurs non-francophones, cela éviterait de devoir le commenter 
à deux reprises (une tâche qui s'avèrerait bien plus pénible que de traduire ce document-ci).

\section{Origine du nom "Project AG"}

Le choix du terme "\textit{Project}" s'explique simplement car il s'agit à la base d'un simple 
projet personnel qui s'est enrichi au fil du temps jusqu'à justifier une mise en ligne. Le même 
terme fait également référence à son aspect \textit{open source} et la possibilité que PAG soit 
enrichi au fil du temps.

Les lettres AG constituent un acronyme pour "\textit{Another G...}". Pourquoi donc ? Tout 
simplement car la majorité des sites web et forums consacrés aux jeux vidéo ont un nom qui 
commence en G. Ni plus ni moins ! Il s'agit donc de dire, un peu ironiquement, que PAG n'est 
jamais qu'un \textit{autre site} sur le sujet.

Enfin, \textit{Project AG} en tant que de nom de domaine (c.-à-d. le \texttt{projectag.org}) est 
aussi un jeu de mot. En effet, ce nom de domaine sonne comme "\textit{project tag}" et fait 
allusion à la particularité du forum de classer les sujets par mots-clefs (ou \textit{tags}) 
plutôt que par catégories. Il s'agit par ailleurs d'une des premières fonctionnalités qui a été 
implémentées lors de la création du site.
