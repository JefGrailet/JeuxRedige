\chapter*{Avant-propos}
\setcounter{chapter}{0}
\addcontentsline{toc}{chapter}{Avant-propos}

Project AG est un site web qui a été conçu et créé par pur loisir. Bien qu'il soit actuellement 
axé sur les jeux vidéo, PAG (pour abréger) s'est construit à la base autour d'un forum et de 
l'espace membre qui s'y rattache. En ce sens, son code n'est pas forcément spécifique aux sites 
traitant des jeux vidéo (et autres) et peut être ré-utilisé à d'autres fins.

Comme le code ne contenait rien d'exotique et que le site n'a pas pour vocation de devenir une 
source de revenus, j'ai décidé de le rendre \textit{open source} afin qu'il puisse aussi bien 
servir à d'autres développeurs qu'être amélioré et étendu au fil du temps par des contributeurs 
externes. Ce choix est d'autant plus justifié que le base de données associée au site ne stocke 
et ne traite aucune donnée sensible (à moins de considérer les adresses e-mail comme des données 
sensibles), les mots de passe étant \textit{hachés}\footnote{Cf. 
\url{https://fr.wikipedia.org/wiki/Fonction_de_hachage}}).

Comme PAG a été conçu sur plusieurs années par à-coups, son code est commenté en long et en large 
afin de toujours garder une trace de l'intérêt de chaque composant ainsi que de la manière de 
les utiliser. La lecture du code devrait donc être relativement aisée pour un oeil extérieur, mais 
les commentaires existants ne seront pas forcément suffisants pour comprendre spontanément 
l'architecture du site et le pourquoi du comment de certains choix de conception.

C'est précisément pour donner un aperçu d'ensemble du code que ce document a été créé. Les pages 
qui suivent décrivent donc l'architecture globale du site, en allant du plus général (langages 
utilisés, arborescence des dossiers, etc.) vers le plus particulier (composition typique d'un 
script principal, gestion de l'affichage).
