\chapter{Le \textit{Modèle} et la base de données}\label{model}

% Cette 
% idée présente deux avantages: d'une part, les requêtes SQL n'apparaissent jamais directement 
% dans les scripts manipulant les éléments du site, et d'autre part, toutes les requêtes liées à un 
% élément en particulier sont rassemblées dans une ou deux classes. Il est dès lors facile de les 
% retrouver en cas de problème, ou de les modifier sans toucher au reste du site du moment que 
% l'interface de la classe n'est pas modifiée.
